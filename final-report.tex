% This is samplepaper.tex, a sample chapter demonstrating the
% LLNCS macro package for Springer Computer Science proceedings;
% Version 2.21 of 2022/01/12
%
\documentclass[runningheads]{llncs}

\usepackage[utf8]{inputenc}
\usepackage{amsmath}
\usepackage{geometry}
\geometry{a4paper, left=25mm, right=25mm, top=25mm, bottom=25mm}

% \usepackage{fontspec}
%
\usepackage[T1]{fontenc}
% T1 fonts will be used to generate the final print and online PDFs,
% so please use T1 fonts in your manuscript whenever possible.
% Other font encondings may result in incorrect characters.
%
\usepackage{graphicx}
% Used for displaying a sample figure. If possible, figure files should
% be included in EPS format.
%
% If you use the hyperref package, please uncomment the following two lines
% to display URLs in blue roman font according to Springer's eBook style:
%\usepackage{color}
%\renewcommand\UrlFont{\color{blue}\rmfamily}
%


\begin{document}

\begin{titlepage}
    \centering
    \vspace*{2cm}
    
    {\scshape\Large PhD Programme in Computer Science and Engineering \par}
    \vspace{0.5cm}
    {\scshape\large Cycle XXXIX \par}
    \vspace{0.5cm}

    \rule{\linewidth}{0.4mm} \\ [0.1mm]
    \raisebox{0.2cm}{\rule{\linewidth}{0.8mm}} \\[0.8cm]
    {\huge\bfseries PhD First Year -- Final Report \par}
    \vspace{0.8cm}
    \rule{\linewidth}{0.8mm} \\ [0.1pt]
    \raisebox{0.2cm}{\rule{\linewidth}{0.4mm}} \\[1.5cm]
    
    % {\Large PhD First Year -- Final Report \par}

    \vspace{1.5cm}
    
    \noindent
    \begin{minipage}[t]{0.45\textwidth}
        \raggedright
        \textbf{Commission:}\\[0.5cm]
        Prof. Mirko Viroli\\
        Prof. Danilo Pianini\\
        Prof. Matteo Ferrara
    \end{minipage}%
    \hfill
    \begin{minipage}[t]{0.45\textwidth}
        \raggedleft
        \textbf{PhD Student:}\\[0.5cm]
        Davide Domini
    \end{minipage}
    
    \vfill
    
\end{titlepage}


\section{Pubblications}
List of published (or already accepted) papers:
\begin{enumerate}
    \item \emph{ScaRLib: A Framework for Cooperative Many Agent Deep Reinforcement Learning in Scala}~\cite{DBLP:conf/coordination/DominiCAV23} \\
    \textbf{Abstract: }
    Multi Agent Reinforcement Learning (MARL) is an emerging field in machine learning where multiple agents learn, simultaneously and in a shared environment, 
     how to optimise a global or local reward signal. MARL has gained significant interest in recent years due to its successful applications in various domains,
     such as robotics, IoT, and traffic control. Cooperative Many Agent Reinforcement Learning (CMARL) is a relevant subclass of MARL, where thousands of 
     agents work together to achieve a common coordination goal.
    %
    In this paper, we introduce ScaRLib, a Scala framework relying on state-of-the-art deep learning libraries to support the development of CMARL systems. 
    %
    The framework supports the specification of centralised training and decentralised execution, and it is designed to be easily extensible, allowing to add new 
     algorithms, new types of environments, and new coordination toolchains.
    %
    This paper describes the main structure and features of ScaRLib and includes basic demonstrations that showcase binding with one such toolchain: 
     ScaFi programming framework and Alchemist simulator can be exploited to enable learning of field-based coordination policies for large-scale systems.
    %
    \item \emph{Field-Based Coordination for Federated Learning}~\cite{DBLP:conf/coordination/DominiAEV24} \\
    \textbf{Abstract: }
    Federated Learning has gained increasing interest in the last years, as it allows the training of machine learning models with a large number of devices 
     by exchanging only the weights of the trained neural networks. 
    % 
    Without the need to upload the training data to a central server, privacy concerns and potential bottlenecks can be removed as fewer data is transmitted. 
    %
    However, the current state-of-the-art solutions are typically centralized, and do not provide for suitable coordination mechanisms to take into account 
     spatial distribution of devices and local communications, which can sometimes play a crucial role. 
    % 
    Therefore, we propose a field-based coordination approach for federated learning, where the devices coordinate with each other through the use of 
     computational fields. 
    %
    We show that this approach can be used to train models in a completely peer-to-peer fashion. 
    %
    Additionally, our approach also allows for emergently create zones of interests, and produce specialized models for each zone enabling each agent 
     to refine their models for the tasks at hand.
    %
    We evaluate our approach in a simulated environment leveraging aggregate computing—the reference global-to-local field-based coordination programming paradigm. 
    %
    The results show that our approach is comparable to the state-of-the-art centralized solutions, while enabling a more flexible and scalable approach 
     to federated learning.
    \item \emph{ScaRLib: Towards a hybrid toolchain for aggregate computing and many-agent reinforcement learning}~\cite{DBLP:journals/scp/DominiCAV24} \\
    \textbf{Abstract: }
    This article introduces ScaRLib, a Scala-based framework that aims to streamline the development cyber-physical swarms scenarios 
     (i.e., systems of many interacting distributed devices that collectively accomplish system-wide tasks) by integrating macroprogramming and multi-agent 
     reinforcement learning to design collective behavior. 
    %
     This framework serves as the starting point for a broader toolchain that will integrate these two approaches at multiple points to harness the 
     capabilities of both, enabling the expression of complex and adaptive collective behavior.

    \item \emph{Towards Intelligent Pulverized Systems: a Modern Approach for Edge-Cloud Services}~\cite{DBLP:conf/woa/DominiFAV24} \\
    \textbf{Abstract: }
    Emerging trends are leveraging the potential of the edge-cloud continuum to foster the creation of smart services capable of adapting to the 
     dynamic nature of modern computing landscapes. 
    %
     This adaptation is achievable through two primary methods: by leveraging the underlying architecture to refine machine learning algorithms, 
      and by implementing machine learning algorithms to optimize the distribution of resources and services intelligently. 
    %  
    This paper explores the latter approach, focusing on recent advancements in pulverized architecture, collective intelligence, and many-agent 
     reinforcement learning systems. 
    %
    This novel trend, which we refer to as intelligent pulverized system (IPS), aims to create a new generation of services that can adapt to the 
     complex and dynamic nature of the edge-cloud continuum. Our proposed learning framework integrates many-agent reinforcement learning, graph neural 
     networks, and aggregate computing to create intelligent services tailored for this environment. 
    %
    We discuss the application of this framework across different levels of the pulverization model, illustrating its potential to enhance 
     the adaptability and efficiency of services within the edge-cloud continuum.
    \item \emph{Proximity-based Self-Federated Learning}~\cite{DBLP:conf/acsos/DominiFAVE24} \\
    \textbf{Abstract: }
    In recent advancements in machine learning, federated learning allows a network of distributed clients to collaboratively develop a global model 
     without needing to share their local data. 
    %
    This technique aims to safeguard privacy, countering the vulnerabilities of conventional centralized learning methods. 
    %
    Traditional federated learning approaches often rely on a central server to coordinate model training across clients, aiming to replicate 
     the same model uniformly across all nodes. 
    %
    However, these methods overlook the significance of geographical and local data variances in vast networks, potentially affecting model 
     effectiveness and applicability. 
    % 
    Moreover, relying on a central server might become a bottleneck in large networks, such as the ones promoted by edge computing. 
    %
    Our paper introduces a novel, fully-distributed federated learning strategy called proximity-based self-federated learning that enables 
     the self-organised creation of multiple federations of clients based on their geographic proximity and data distribution without 
     exchanging raw data.
    %
    Indeed, unlike traditional algorithms, our approach encourages clients to share and adjust their models with neighbouring nodes based on geographic 
     proximity and model accuracy. 
    %
    This method not only addresses the limitations posed by diverse data distributions but also enhances the model's adaptability to different regional 
     characteristics creating specialized models for each federation. We demonstrate the efficacy of our approach through simulations on 
     well-known datasets, showcasing its effectiveness over the conventional centralized federated learning framework.

    \item \emph{Towards Self-Adaptive Cooperative Learning in Collective Systems}~\cite{DBLP:conf/acsos/Domini24} \\
    \textbf{Abstract: }
    Nowadays, collective adaptive systems have become crucial as modern systems increasingly adopt this vision.
    %
    These systems can be leveraged as a means to facilitate cooperative adaptive learning.
    %
    However, implementing such systems presents various challenges, including:
     scalability, failures, non-iid data and complex architectures.
    %
    This paper presents a modern approach to cooperative and privacy-resilient learning by leveraging macroprogramming.
    %
    Specifically, we propose a new framework based on the integration of aggregate computing and federated learning,
     aiming to address these challenges and enhance the effectiveness and security of cooperative learning systems.

    \item \emph{A Reusable Simulation Pipeline for Many-Agent Reinforcement Learning}~\cite{DBLP:conf/dsrt/DominiAPV24} \\
    \textbf{Abstract: }
    Recent advancements in multi-agent reinforcement learning led to systems in which large groups of agents
     work together to learn shared policies and achieve collective behavior.
    %
    This approach is increasingly important for many applications,
     including swarm robotics, crowd sensing, and large-scale IoT networks.
    %
    In fact, these systems require repeated experimentation to learn from experience:
     simulation becomes thus essential, as deploying and testing in real-world environments incurs in high costs and practical challenges.
    %
    In response to this need, our paper introduces a simulation-based pipeline to gather the necessary experience for many-agent learning.
    %
    We highlight the requirements of such pipeline and the role of simulation, presenting also a practical prototype implemented in Alchemist,
     a simulator designed for very large-scale systems.
    %
    This pipeline provides a scalable, modular, and flexible environment for developing and testing many-agent reinforcement learning strategies.
\end{enumerate}

\section{Attended Conferences and Workshops}
\begin{enumerate}
    \item \textbf{Coordination Models and Languages} - 26th International Conference, COORDINATION 2024, Held as Part of the 19th International Federated Conference on Distributed Computing Techniques, DisCoTec 2024, Groningen, The Netherlands, June 17-21, 2024
    \item \textbf{25th Workshop "From Objects to Agents"}, Bard (Aosta), Italy, July 8-10, 2024
    \item \textbf{IEEE International Conference on Autonomic Computing and Self-Organizing Systems}, ACSOS 2024, Aarhus, Denmark, September 16-20 2024
    \item \textbf{11th Workshop on Self-Improving Systems Integration}, SISSY 2024, Held as Part of ACSOS 2024, Aarhus, Denmark, September 16-20 2024
    \item \textbf{2nd International Workshop on Artificial Intelligence for Autonomous computing Systems}, AI4AS 2024, Held as Part of ACSOS 2024, Aarhus, Denmark, September 16-20 2024
    \item \textbf{28th International Symposium on Distributed Simulation and Real Time Applications}, DS-RT 2024, Urbino, Italy, October 7-9 2024
\end{enumerate}

\section{Doctoral Schools}
\begin{enumerate}
    \item \textbf{Bertionoro International Spring School 2024}, BISS 2024, Bertinoro, Italy, March 11-15 ,2024
\end{enumerate}

\section{Teaching Tutor}
\begin{enumerate}
    \item \textbf{Machine Learning Systems for Data Science} - Statistical Science Bachelor Degree
    \item \textbf{Software Engineering (Modulo 1)} - Digital Transformation Management Master Degree
\end{enumerate}

\section{PhD Courses}

\begin{enumerate}
    \item \textbf{BISS 2024 - Bertinoro International Spring School - Program Analysis}
    \begin{itemize}
        \item Prof: Roberta Gori, Roberto Bruni  - Università di Pisa
        \item Proposed CFU: 4 (12 hours) 
        \item Period: March 2024
        \item Exam: Not done, CFUs without evaluation
    \end{itemize}    
    \item \textbf{BISS 2024 - Bertinoro International Spring School - Graph Neural Networks}
    \begin{itemize}
        \item Prof: Fabrizio Silvestri - Sapienza Università di Roma
        \item Proposed CFU: 4 (12 hours) 
        \item Period: March 2024
        \item Exam: Done
    \end{itemize}  
    \item \textbf{BISS 2024 - Bertinoro International Spring School - Large Language Models}
    \begin{itemize}
        \item Prof: Danilo Croce - University of Roma, Tor Vergata
        \item Proposed CFU: 4 (12 hours) 
        \item Period: March 2024
        \item Exam: Done
    \end{itemize}  
    \item \textbf{Risk Assessment of ML for Cybersecurity}
    \begin{itemize}
        \item Fabio Pierazzi - King's College London
        \item Proposed CFU: 5 (20 hours) 
        \item Period: April 2024
        \item Exam: Done
    \end{itemize}    
    \item \textbf{Introduction to Complex Systems Science} 
    \begin{itemize}
        \item Prof: Andrea Roli - Università di Bologna      
        \item Proposed CFU: 2 (10 hours) 
        \item Period: June 2024
        \item Exam: To do
    \end{itemize}    
    
\end{enumerate}
%
% ---- Bibliography ----
%
% BibTeX users should specify bibliography style 'splncs04'.
% References will then be sorted and formatted in the correct style.
%
\bibliographystyle{alpha}
\bibliography{mybibliography}
%
% \begin{thebibliography}{8}
% \bibitem{ref_article1}
% Author, F.: Article title. Journal \textbf{2}(5), 99--110 (2016)

% \bibitem{ref_lncs1}
% Author, F., Author, S.: Title of a proceedings paper. In: Editor,
% F., Editor, S. (eds.) CONFERENCE 2016, LNCS, vol. 9999, pp. 1--13.
% Springer, Heidelberg (2016). \doi{10.10007/1234567890}

% \bibitem{ref_book1}
% Author, F., Author, S., Author, T.: Book title. 2nd edn. Publisher,
% Location (1999)

% \bibitem{ref_proc1}
% Author, A.-B.: Contribution title. In: 9th International Proceedings
% on Proceedings, pp. 1--2. Publisher, Location (2010)

% \bibitem{ref_url1}
% LNCS Homepage, \url{http://www.springer.com/lncs}. Last accessed 4
% Oct 2017
% \end{thebibliography}
\end{document}
